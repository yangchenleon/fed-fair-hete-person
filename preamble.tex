\usepackage{ctex} % 中文支持
\usepackage{graphicx} % 插入图片
\usepackage{amsmath, amssymb} % 数学符号/公式
\usepackage{hyperref} % 超链接,用在参考文献
\usepackage{acro} % 管理缩写词表
\usepackage{fontspec} % 字体 use xelatex
\usepackage{color} % 颜色
\usepackage[left=2.50cm,right=2.50cm,top=2.80cm,bottom=2.50cm]{geometry}% 页边距设置A4

\renewcommand{\baselinestretch}{1.5} % 定义行间距(1.5)
\newcommand{\myref}[1]{Eq.\ref{#1}} % 引用公式带有Eq

\graphicspath{{figures/}}
\makeatletter
\def\input@path{{contents/}}
\makeatother

\bibliographystyle{unsrt} % plain 按照作者顺序 unsrt按照引用顺序

\DeclareAcronym{iot}{short=IoT, long=Internet of Things}
\DeclareAcronym{ec}{short=EC, long=Edge Computing}
\DeclareAcronym{mec}{short=MEC, long=Mobile Edge Computing}
\DeclareAcronym{iid}{short=I.I.D., long=Independent and Identically Distributed}
\DeclareAcronym{niid}{short=Non-I.I.D., long=Non Independent and Identically Distributed}
\DeclareAcronym{fl}{short=FL, long=Federated Learning}
\DeclareAcronym{pfl}{short=PFL, long=Personalized Federated Learning}
\DeclareAcronym{hfl}{short=HFL, long=Heterogenous Federated Learning}
\DeclareAcronym{kd}{short=KD, long=Knowledge Distillation}
\DeclareAcronym{kl}{short=KL, long=Kullback–Leibler}
\DeclareAcronym{ue}{short=UE, long=User Equipment}

\title{A more fair and efficient Heterogenous / Personalized Federated Learning framework}
\author{YCC}
\date{\today}
